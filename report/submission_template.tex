% ==============================================================
%  PROJECT REPORT TEMPLATE (LIGHT VERSION)
%  Who Is The Killer? — Piraeus Vice
% ==============================================================

\documentclass[11pt,a4paper]{article}

\usepackage[utf8]{inputenc}
\usepackage[T1]{fontenc}
\usepackage[english]{babel}
\usepackage{amsmath,amssymb}
\usepackage{geometry}
\geometry{margin=2.5cm}
\usepackage{graphicx}
\usepackage{booktabs}
\usepackage{enumitem}
\usepackage{hyperref}
\usepackage{mwe} % for example-image, example-image-duck, etc.

\title{
  Who Is The Killer?\\[0.3em]
  {\large Piraeus Vice Pattern Recognition Project}
}

\author{
  Student 1 Name (ID: 0000001) \\
  Student 2 Name (ID: 0000002) \\
  Student 3 Name (ID: 0000003) \\
  \\[-0.3em]
  Department of \dots \\
  University of \dots
}

\date{\today}

\begin{document}

\maketitle

\begin{abstract}
Short summary of your approach and main findings.
Mention the models you used (Bayes classifier, linear classifier, SVM,
MLP, PCA, $k$-means) and the overall performance on the validation data.
\end{abstract}

\tableofcontents
\newpage

% --------------------------------------------------------------
\section{Introduction}

Briefly describe the problem: identifying the most likely killer for
each crime incident in the ``Piraeus Vice'' data set. Summarise the data
structure at a high level and list the main steps of your analysis
(Q1--Q8).

\begin{figure}[h]
  \centering
  \includegraphics[width=0.55\textwidth]{example-image-duck}
  \caption{Humorous placeholder: the ``prime suspect'' before any
  machine learning. Replace this with a real illustrative figure (e.g.,
  a high-level pipeline diagram of your method).}
\end{figure}

% --------------------------------------------------------------
\section{Data description}

Describe the dataset:
TRAIN / VAL / TEST splits, number of incidents, continuous and
categorical features, any preprocessing (standardisation, one-hot
encoding).

\subsection{Feature overview}

You may reproduce a small table summarising the features:

\begin{table}[h]
  \centering
  \caption{Example feature summary (replace with real one).}
  \begin{tabular}{ll}
    \toprule
    Feature name       & Type \\
    \midrule
    hour\_float        & continuous \\
    latitude           & continuous \\
    longitude          & continuous \\
    victim\_age        & continuous \\
    weapon\_code       & categorical \\
    scene\_type        & categorical \\
    weather            & categorical \\
    vic\_gender        & categorical \\
    \bottomrule
  \end{tabular}
\end{table}

% --------------------------------------------------------------
\section{Q1: Exploratory analysis}

Include histograms or density plots for key continuous variables
(hour, age, latitude, longitude). Compare a single Gaussian fit and a
mixture of Gaussians for \texttt{hour\_float}.

\begin{figure}[h]
  \centering
  \includegraphics[width=0.7\textwidth]{example-image}
  \caption{Placeholder: histogram and fitted curves for an example
  feature (e.g.\ \texttt{hour\_float}). Replace with your actual
  exploratory plot.}
\end{figure}

% --------------------------------------------------------------
\section{Q2: Gaussian MLE per killer}

Explain how you estimated the mean and covariance of the continuous
features for each killer on TRAIN. Show at least one covariance heatmap
and one 2D projection with ellipses per killer.

\begin{figure}[h]
  \centering
  \includegraphics[width=0.6\textwidth]{example-image-b}
  \caption{Placeholder: covariance ``heatmap'' for a single killer.
  Replace with your real covariance heatmap for the continuous features.}
\end{figure}

% --------------------------------------------------------------
\section{Q3: Multiclass Gaussian Bayes classifier}

Describe your implementation of the Gaussian Bayes classifier using the
parameters from Q2. Report accuracy and a confusion matrix on the VAL
split.

\begin{table}[h]
  \centering
  \caption{Example confusion matrix for the Gaussian Bayes classifier
  on VAL (numbers are illustrative only).}
  \begin{tabular}{lcccc}
    \toprule
    True / Pred & 1 & 2 & 3 & 4 \\
    \midrule
    1           & 30 &  2 &  1 &  0 \\
    2           &  3 & 25 &  4 &  1 \\
    3           &  0 &  3 & 28 &  2 \\
    4           &  1 &  0 &  2 & 29 \\
    \bottomrule
  \end{tabular}
\end{table}

% --------------------------------------------------------------
\section{Q4: Linear classifier}

Explain the linear model you used (linear network or softmax
regression), the features, and training procedure. Report validation
accuracy and confusion matrix; briefly compare with Q3.

\begin{figure}[h]
  \centering
  \includegraphics[width=0.6\textwidth]{example-image-c}
  \caption{Placeholder: decision regions of a linear classifier in a
  2D PCA projection. In the final report show your own PCA plot with
  approximate linear boundaries.}
\end{figure}

% --------------------------------------------------------------
\section{Q5: Support Vector Machine}

Describe the SVM setup (kernel, hyperparameter tuning, multiclass
strategy). Report VAL accuracy and confusion matrix, and compare with
Q3 and Q4.

\begin{table}[h]
  \centering
  \caption{Example summary of validation accuracy for different models.}
  \begin{tabular}{lcc}
    \toprule
    Model                  & VAL accuracy \\
    \midrule
    Gaussian Bayes         & 0.75 \\
    Linear classifier      & 0.70 \\
    SVM (RBF kernel)       & 0.81 \\
    \bottomrule
  \end{tabular}
\end{table}

% --------------------------------------------------------------
\section{Q6: Multi-Layer Perceptron}

Summarise your MLP architecture (layers, activations), training
procedure, and performance on VAL. Include a simple plot or table of
feature importance (e.g.\ permutation importance).

\begin{figure}[h]
  \centering
  \includegraphics[width=0.7\textwidth]{example-image}
  \caption{Placeholder: bar plot of feature importance as estimated by
  the MLP. Replace with your own figure showing which features matter
  most for killer prediction.}
\end{figure}

% --------------------------------------------------------------
\section{Q7: Principal Component Analysis}

Describe how you applied PCA (preprocessing, number of components kept).
Include a plot of eigenvalues versus component index and a PC1--PC2
scatter plot coloured by predicted killer labels.

\begin{figure}[h]
  \centering
  \includegraphics[width=0.7\textwidth]{example-image-a}
  \caption{Placeholder: ``eigenvalues vs component index'' plot. In
  practice this shows how much variance each principal component
  explains.}
\end{figure}

% --------------------------------------------------------------
\section{Q8: $k$-means in PCA space}

Explain how you ran $k$-means on the PCA representation, how you mapped
clusters to killers using majority vote on TRAIN, and how you evaluated
accuracy on VAL. Show a PC1--PC2 scatter plot of TEST incidents coloured
by your final predicted killer labels.

\begin{figure}[h]
  \centering
  \includegraphics[width=0.7\textwidth]{example-image-duck}
  \caption{Placeholder: clusters in 2D PCA space, coloured by predicted
  killer. Ideally, your real figure will look less like ducks and more
  like well-separated clusters.}
\end{figure}

% --------------------------------------------------------------
\section{Discussion and conclusions}

Summarise and compare the performance of all models (Bayes, linear,
SVM, MLP, PCA+$k$-means). Discuss which features and methods were most
helpful for identifying killers, and reflect on possible extensions
(e.g.\ more complex generative models, temporal information).

% --------------------------------------------------------------
\section*{Code organisation}

Briefly describe your Python code structure (scripts, notebooks, helper
modules) and how to reproduce:
\begin{itemize}
  \item the main results and figures in this report;
  \item the prediction file \texttt{submission.csv}.
\end{itemize}

% --------------------------------------------------------------
\appendix

\section{Additional figures and tables}

Place any extra plots or tables here that do not fit into the main text.

\section{LLM prompts and responses}

If you used ChatGPT or any other LLM, list all prompts and the
corresponding responses you used during the preparation of this project.

\subsection*{Example format}

\textbf{Prompt:}
\begin{verbatim}
Explain how to compute a covariance matrix from data.
\end{verbatim}

\textbf{Response:}
\begin{verbatim}
[Paste the model's response here...]
\end{verbatim}

\textbf{Comment:} We checked the formulas and then implemented them
independently.

\end{document}
